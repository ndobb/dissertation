
\documentclass[../main.tex]{subfiles}

\begin{document}

\noindent\textbf{Objective:} Finding patients within clinical databases meeting certain criteria is a critical step for clinical trials and biomedical research. Yet accurate query generation often requires extensive technical- and biomedical domain-specific expertise and knowledge of data models and terminologies. We sought to create a system capable of generating data model agnostic queries while also providing novel logical reasoning capabilities for complex clinical trial eligibility criteria. \\

\noindent\textbf{Materials and Methods:} The task of query generation from eligibility criteria requires solving several text processing problems, including named entity recognition and relation extraction, sequence-to-sequence transformation, normalization, and reasoning. We incorporated hybrid deep learning and rule-based modules for these, as well as a knowledge base of the UMLS and linked ontologies. To enable query generation agnostic to data models, we introduce a novel method for tagging database schema elements using UMLS concepts. To evaluate our system, called LeafAI, we compared patients found by queries generated by LeafAI and a human database programmer to patients enrolled in 8 past clinical trials at our institution. We measured performance of each by the number of actual enrolled patients matched by generated queries. \\

\noindent\textbf{Results:} LeafAI matched a mean 43\% of enrolled patients across 8 clinical trials, compared to 27\% matched in queries by a human database programmer. \\

\noindent\textbf{Conclusions:} Our work contributes a state-of-the-art data model-agnostic query generation system capable of conditional reasoning using a knowledge base. We demonstrate that results of patients found by the LeafAI query engine can rival that of a human programmer in finding patients eligible for clinical trials.

\end{document}