
\documentclass[../main.tex]{subfiles}

\begin{document}

\noindent\textbf{Objective:} Finding patients within clinical databases meeting certain criteria is a critical step for clinical trials and biomedical research. Yet accurate query generation often requires extensive technical- and biomedical domain-specific expertise and knowledge of data models and terminologies. We sought to create a system capable of generating data model-agnostic queries while also providing novel logical reasoning capabilities for complex clinical trial eligibility criteria. \\

\noindent\textbf{Materials and Methods:} The LeafAI query engine utilizes a micro-service-based pipeline approach for query generation using named entity recognition and relation extraction, sequence to sequence transformation into logical intermediate representations, normalization, reasoning using a knowledge base of the UMLS and linked ontologies, and finally query generation using database schema semantic metadata. To evaluate our system, we compared patients found by queries generated by LeafAI and a human database programmer to patients enrolled in actual clinical trials at our institution. We measured each system by the number of actual enrolled patients matched in their queries. \\

\noindent\textbf{Results:} LeafAI matched a mean 43.5\% of enrolled patients across 8 clinical trials, compared to 27.2\% matched in queries by a human database programmer. \\

\noindent\textbf{Conclusions:} Our work contributes a state-of-the-art data model-agnostic query generation system capable of conditional reasoning using a knowledge base. We demonstrate that results of patients found by the LeafAI query engine can rival that of a human programmer in finding patients eligible for clinical trials.

\end{document}