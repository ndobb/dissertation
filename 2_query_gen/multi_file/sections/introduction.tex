
\documentclass[../main.tex]{subfiles}

\begin{document}

Identifying groups of patients meeting a given set of eligibility criteria is a critical step for recruitment into randomized controlled trials (RCTs). Yet many trials fall short of recruitment goals, leading to time and cost overruns while creating challenges in ensuring adequate statistical power \cite{gul2010clinical, adams2015barriers}. Failures in adequate recruitment may result from a variety of factors, but can often stem from difficulties in translating complex eligibility criteria into queries using data collected in the electronic medical record (EHR) \cite{wang2017classifying}. Despite these difficulties, RCTs increasingly rely on EHR data as a useful and more expedient means of identifying potential patients versus manual chart or case report form review \cite{cowie2017electronic}. At the same time, EHRs increasingly capture and store patient health and outcomes data in greater volume and variety, creating additional challenges and opportunities for patient recruitment \cite{lee2017medical}. While more granular - and potentially useful - data may be captured and stored in EHRs than in the past, the process of accessing and leveraging those data often require extensive technical expertise and knowledge of biomedical terminologies and data models. 

Cohort discovery tools such as Leaf \cite{dobbins2019leaf} and i2b2 \cite{murphy2010serving} may be used in many cases, as they offer relatively simple drag-and-drop interfaces capable of querying EHR data to find patients meeting given criteria \cite{johnson2014use}. Yet these tools are not a panacea, as they often have significant learning curves and may be unable to represent particularly complex nested or temporal eligibility criteria \cite{deshmukh2009evaluating}. Moreover, existing cohort discovery tools lack functionality to dynamically reason upon non-specific criteria that frequently appear in real-world eligibility criteria. For example, a criterion may require patients "indicated for bariatric surgery", but translating such non-specific criteria into a query (e.g., patients with a diagnosis of morbid obesity or body mass index greater than 40) must be performed manually by a researcher, even in cases where constructing an exhaustive list of such criteria may be time-intensive, subjective, and error-prone. 

In recent years, alternatives to web-based cohort discovery tools have been explored. In particular, various methods using Natural Language Processing (NLP) have been put forth by the research community \cite{yuan2019criteria2query, soni2020patient, fang2022combining, zhang2020deepenroll, chen2019clinical, patrao2015recruit, dhayne2021emr2vec, liu2021evaluating, xiong2019cohort}. NLP-based cohort discovery methods hold unique potential and appeal, as they are hypothetically able to leverage existing eligibility criteria described in natural language, a medium researchers and investigators already use and are comfortable with.

\end{document}