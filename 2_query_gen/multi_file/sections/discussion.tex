
\documentclass[../main.tex]{subfiles}

\begin{document}

Our results demonstrate that under certain circumstances, LeafAI is capable of rivaling the ability of a human programmer in matching patients eligible to clinical trials. Indeed, in numerous cases we found LeafAI and the human programmer executing similar queries, such as for Hepatitis C (NCT04852822), Chronic Lymphocytic Leukemia (NCT04852822), Multiple Sclerosis (NCT03621761), and Diabetes Mellitus (NCT03029611), where both ultimately matched a similar number of patients.

One notable pattern we found is that LeafAI consistently finds a higher number of potentially eligible patients. As we have not done manual chart review of the patients found, it is difficult to determine the proportion of true positive versus false negative eligible patients compared to eligible patients found by the human programmer. We hypothesize that in many cases, LeafAI's Knowledge Base played a key role both finding additional eligible patients, but also sometimes in unnecessarily excluding otherwise eligible patients. For example, in the Multiple Sclerosis (MS) trial, LeafAI searched for 11 different SNOMED codes related to MS (including MS of the spinal cord, MS of the brain stem, acute relapsing MS, etc.), while the human programmer searched for only one, and ultimately LeafAI found nearly 5 times the number of potentially eligible patients (4,891 versus 1,016). We hypothesize that the the human programmer likely had a lower rate of false positives (and thus higher precision), though we leave an analysis of that to future work. On the other hand, in the same trial, as can be seen in \ref{fig_leafai_results_analysis}, given the exclusion criteria: "Current shift work sleep disorder, or narcolepsy diagnosed with polysomnography and multiple sleep latency", LeafAI's KB included diagnosis codes for drowsiness, snoring, and so on, as within the UMLS they appear as child concepts of sleep disorder (C0851578). The exclusion of these patients likely resulted in an approximately 40\% drop in recall at that stage compared to the human programmer, though ultimately both achieved similar recall (39\% versus 35\%).

Beyond performance as measured by recall, it is notable that in total, the human programmer spent approximately 26 hours in crafting queries for the 8 trials while LeafAI took to only several minutes running on a single laptop. We believe that the time saved by used automated means such as LeafAI for cohort discovery may save health organizations significant time and resources.

\subsection*{Limitations}

This project has a number of limitations. First, while the 8 clinical trials we evaluated were randomly chosen, we specifically restricted the categories of diseases to select trials among, and thus our results should not be presumed to generalize to other kinds of clinical trial. Next, we evaluated our queries using an OMOP-based extract of data from our EHR. Our OMOP database does not contain the full breadth of data within our EHR. Had our experiments instead been conducted using data directly from our enterprise data warehouse (populated by our EHR), it is very possible the human programmer would have achieved greater results than LeafAI due to knowledge and experience using granular source data. For example, in the Cardiac Arrest clinical trial, the human programmer noted that data for use of cooling blankets is available in our EHR, but not in OMOP. LeafAI would likely not have been able to easily utilize that data, while the human perhaps would have been. 

\subsection*{Future work}

We are actively developing a web-based user interface for LeafAI, shown in Figure \ref{fig_leafai_screenshot}. In future work, we will deploy a prototype of the tool and evaluate user feedback and system performance. The LeafAI web application will provide rapid feedback to users explaining its search strategies, and allow users to override system-reasoned concepts and edit or add their own. Additionally, we intend to explore the adaptation of our logical form-based query generation methods to general-purpose question answering.

\begin{figure}[H]
  \includegraphics[scale=0.26]{figures/leafai_screenshot.png}  
\caption{Example screenshot of the LeafAI web application, which is currently in development.}
\label{fig_leafai_screenshot}
\end{figure}

\end{document}