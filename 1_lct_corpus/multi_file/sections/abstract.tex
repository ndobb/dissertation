
\documentclass[../main.tex]{subfiles}

\begin{document}
\noindent
Identifying cohorts of patients based on eligibility criteria such as medical conditions, procedures, and medication use is critical to recruitment for clinical trials. Such criteria are often most naturally described in free-text, using language familiar to clinicians and researchers. In order to identify potential participants at scale, these criteria must first be translated into queries on clinical databases, which can be labor-intensive and error-prone. Natural language processing (NLP) methods offer a potential means of such conversion into database queries automatically. However they must first be trained and evaluated using corpora which capture clinical trials criteria in sufficient detail. In this paper, we introduce the Leaf Clinical Trials (LCT) corpus, a human-annotated corpus of over 1,000 clinical trial eligibility criteria descriptions using highly granular structured labels capturing a range of biomedical phenomena. We provide details of our schema, annotation process, corpus quality, and statistics. Additionally, we present baseline information extraction results on this corpus as benchmarks for future work.
\end{document}