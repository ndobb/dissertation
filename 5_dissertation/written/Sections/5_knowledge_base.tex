\documentclass[../main.tex]{subfiles}

\begin{document}

\section{Overview}


\section{Related Work}


\section{Methods}

For reasoning and derivation of ICD-10, LOINC, and other codes for UMLS concepts, we designed a KB accessible via SPARQL queries and stored as Resource Description Framework (RDF) \cite{manola2004rdf} triples. The core of our KB is the UMLS, derived using a variation of techniques created for ontologies in BioPortal \cite{noy2009bioportal}. To further augment the UMLS, we mapped and integrated the Disease Ontology \cite{schriml2012disease}, Symptom Ontology \cite{sayers2010database}, COVID-19 Ontology \cite{sargsyan2020covid}, Potential Drug-Drug Interactions \cite{ayvaz2015toward}, LOINC2HPO \cite{zhang2019semantic}, and the Disease-Symptom Knowledge Base \cite{wang2008automated}. We then developed SPARQL queries parameterized by UMLS concepts for various scenarios which leveraged our KB, such as contraindications to treatments, symptoms of diseases, and so on. Using LOINC2HPO mappings further allows us to infer phenotypes by lab test results rather than using ICD-10 or SNOMED codes alone. 

Our KB, nested logical forms, and inside-to-outside normalization methods enable "multi-hop" reasoning on eligibility criteria over several steps. For example, given the non-specific criterion "Contraindications to drugs for conditions which affect respiratory function", our system successfully reasons that (among other results),

\begin{enumerate}
    \item \textbf{Asthma} causes changes to \textbf{respiratory function}
    \item \textbf{Methylprednisolone} can be used to treat \textbf{asthma}
    \item \textbf{Mycosis} (fungal infection) is a contraindication to \textbf{methylprednisolone}
\end{enumerate}

\noindent These features allow LeafAI to reason upon fairly complex non-specific criteria.



\end{document}