\documentclass[../main.tex]{subfiles}

\begin{document}

\subsection{The Importance of Randomized Clinical Trials}

A clinical trial is a prospective study comparing the effects and value of an intervention (typically a medication, biologic, or procedure) against a control group without it \cite{friedman2015fundamentals}. Clinical trials are "randomized" when participants are randomly placed in treatment or control groups. Randomization is considered ideal for reducing risk of investigator bias and producing study groups closely in proportion to known risk factors. Randomized clinical trials (RCTs) are widely recognized as the best available method to determine if a given intervention is safe and effective \cite{friedman2015fundamentals}.

Eligibility for a clinical trial is determined by a trial's \textit{eligibility criteria}, which are free-text descriptions of required conditions, treatments, laboratory tests and so on. Eligibility criteria are typically composed of \textit{inclusions}, which patients \textit{must meet}, and \textit{exclusions}, which patients \textit{must not meet} in order to be eligible. The extent to which trial results can be assumed to be generalizable to patients not participating in a trial but with comparable health status is determined by its \textit{external validity} \cite{rothwell2005external}. External validity can be influenced by a number of factors, first and foremost the selection of patients potentially eligible for a trial. Finding patients appropriately meeting eligibility criteria in as unbiased way as possible is thus critical to drawing meaningful conclusions from clinical trials, and ultimately scientific progress and improvement of human health.

\subsection{Challenges in Recruitment for Clinical Trials}

Many clinical trials fail to meet their expected number of enrollments \cite{frank2004current, grill2010addressing, heller2014strategies, nipp2019overcoming}. The reasons for this are varied, including overly restrictive inclusion criteria \cite{grill2010addressing}, a lack of awareness on the part of patients, particularly in underserved and historically disadvantaged communities \cite{heller2014strategies}, fear or apprehension of medical research due to past abuses \cite{frank2004current}, and uncertainty of risk on the part of providers leading to withheld offers to participate \cite{nipp2019overcoming}. In addition, patients who do agree to participate in clinical trials tend to be wealthier, have greater access to healthcare resources, be members of ethnic majorities, and often unrepresentative of overall populations of patients suffering from a given condition \cite{grill2010addressing, heller2014strategies, nipp2019overcoming, guadagnolo2009involving, penberthy2010automated, holmes2012increasing}. Beyond questions of generalizability, recruitment challenges also cause delays to clinical trials, with an estimate 86\% of trials delayed between 1 and 6 months, and some for even longer \cite{sullivan2004subject, thadani2009electronic}.

These challenges often have severe effects on the outcomes of clinical trials, and to a certain extent new treatments available to patients. These effects can include inadequate statistical analysis of outcomes, cost overruns, extended duration of trials, increased costs of new medications, and as discussed, treatments that potentially do not exhibit expected beneficial outcomes in understudied populations. \cite{easterbrook1992fate, penberthy2010automated, mcdonald2006influences, marks2002using}.

\subsection{The Case for Software in Matching Patients for Clinical Trials}

While computer software and NLP alone can likely not solve many of these challenges, research suggests that in many cases they can add significant value, time- and cost-savings toward trial recruitment \cite{penberthy2010automated, thadani2009electronic}. For example, Thadani \textit{et al} found electronic screening methods to significantly reduce the burden of manual chart review in one study by approximately 81\% \cite{thadani2009electronic}. Examining multiple clinical trials, Penberthy \textit{et al} similarly found up to a 20-fold decrease in staff time spent reviewing eligible patient records by using electronic screening software. More recently, Ni \textit{et al} used a combination of NLP techniques and structured data analysis to screen for potential clinical trial candidates and compared the results to a gold standard data set reviewed by medical doctors. The authors found their highest performing methods achieved an approximate 90\% workload reduction in chart review and 450\% increase in trial screening efficiency \cite{ni2015automated}. 

Thus though the aim of this project is to produce an application capable of general purpose cohort discovery - rather than solely for the purposes of clinical trial recruitment - clinical trials are a meaningful and valuable means by which to \textbf{gather data}, \textbf{evaluate performance}, and \textbf{measure potential real-world impact} of solutions for cohort discovery. Moreover, screening software and NLP have been demonstrated to dramatically improve trial recruitment efficiency in many scenarios.

In terms of data, the website \url{https://clinicaltrials.gov}, maintained by the United States National Library of Medicine, hosts freely accessible descriptions of hundreds of thousands of clinical trials from around the world. Because clinical trials enrollments are in many cases recorded in EHRs (which also include the same patients' clinical data), they also can serve as a uniquely objective means of measuring the effectiveness of NLP systems in matching actual enrolled participants based on eligibility criteria. Put another way, an NLP-based system for matching patients to real-world eligibility criteria should reasonably be expected to find many or most patients enrolled in a given clinical trial - with the assumption that patients enrolled in those trials correctly met the necessary criteria as determined by study investigators. Thus however imperfect (e.g., a lack of diagnosis data for an existing condition may cause certain patients to be inappropriately deemed ineligible), clinical trials are thus well-suited for evaluation of NLP-based cohort discovery systems and thus a focus of much of this project.

\subsection{Challenges in Electronic Screening and NLP in Clinical Trials}

Using NLP to determine patients potentially eligible for a clinical trial has numerous challenges. Consider, for example, a list of eligibility criteria such as: \footnote{Adapted from trial NCT03254875 at \url{https://clinicaltrials.gov/ct2/show/NCT03254875?cond=breast+cancer&draw=2&rank=3}} \\

\textit{Inclusion Criteria:}

\begin{enumerate}
    \itemsep0em 
    \item \textit{Newly diagnosed with breast cancer and scheduled for surgery}
    \item \textit{18 years or above}
    \item \textit{Those who experience high psychological stress will enter the RCT whereas those with low stress will be followed in an observational questionnaire study}
\end{enumerate}

\textit{Exclusion Criteria:}

\begin{enumerate}
    \item[4.] \textit{No severe psychiatric disease requiring treatment, e.g., schizophrenia}
\end{enumerate}

\noindent While perhaps appearing deceptively simple, this example demonstrates many of the difficulties of this task. In criterion 1,  "Newly" in "Newly diagnosed with breast cancer", suggests that diagnoses occurring further in the past (though how far is unclear) should not be included. Meanwhile, "surgery" in "scheduled for surgery" likely refers to surgery in relation to breast cancer, though this is not explicitly stated. In criterion 2, "18 years" likely refers to participants' age, but this too is not explicitly stated. Criterion 3, meanwhile, is a description of processes which will take place during the trial, but are not actually eligibility criterion (i.e., participants may be eligible whether their actual stress levels are high or low). In criterion 5, "psychiatric disease" is non-specific and may refer to a large number of unstated conditions, aside from schizophrenia which is given as an example.

Appropriately interpreting the semantics and unstated requirements of these criteria are challenging for NLP systems. A system using \textbf{named entity recognition} (NER) and \textbf{relation extraction}, for example, may correctly determine that "breast cancer" refers to a condition and "surgery" refers to a procedure, but may still fail if "Newly" is not determined to refer to breast cancer or not determined to be equivalent to "first time". In a subsequent step, an NLP system may \textbf{normalize} (i.e., determine an coded representation of a concept, for example an ICD-10 or UMLS code) "breast cancer" incorrectly as "Malignant Neoplasms" (UMLS C0006826) rather than "Malignant neoplasm of breast" (UMLS C0006142). In criterion 3, an NLP system may attempt to limit eligibility to patients with high stress levels, despite the criterion not being a formal restriction as such. In criterion 4, an NLP system may fail to reason that other unstated conditions, such as hysteria or hallucinations, should also be excluded.

Beyond challenges in interpreting eligibility criteria, certain criteria - even if interpreted correctly - may simply be absent or even incorrect in the data source the NLP system queries or analyzes. For example, Eastern Cooperative Oncology (ECOG) performance status scores \cite{sok2019objective} are frequently listed in eligibility criteria but often absent in structured clinical databases.

\textbf{Add citations about interpretability and/or user trust in system too}.

\subsection{Broader Impact}

Despite these challenges, clear opportunities exist for NLP-based systems to improve recruitment rates and efficiencies in clinical trials. This project aims to create a user-friendly, explainable, and generalizable application for cohort discovery 

\end{document}