\documentclass[../main.tex]{subfiles}

\begin{document}

This ultimate aim of this project is to develop an interactive web application for cohort discovery on arbitrary database schema using free-text input to generate queries. There are a number of milestones and necessary steps to make such an application possible. In this section, we discuss past research related to various components of this project. We first examine NLP corpora related to eligibility criteria, which often form the foundation for NLP-based query generation systems. Next, we examine NLP approaches for cohort discovery and query generation in general. Finally, we discuss research in web application user interface design and usability which will inform the design of our application.

\subsection{Corpora for Clinical Trials and Eligibility Criteria}

The NLP tasks involved in transforming eligibility criteria into database queries include \textbf{named entity recognition} (NER) to tag meaningful spans of text as named entities, \textbf{relation extraction} to classify relations between named entities, \textbf{normalization} to map named entities to common coded representations (e.g., ICD-10), \textbf{negation detection} to detect negated statements (e.g., "not hypertensive") and so on. Gold standard corpora quality can thus directly affect performance and the validation of each of these tasks. Such corpora can serve as reliable benchmarks for purposes of comparing NLP methods as well as training data sets. 

A number of past corpora related to clinical trials have been published \cite{weng2011elixr, kang2017eliie, kury2020chia, yu2020}. Weng \textit{et al} developed EliXR \cite{weng2011elixr}, a rule-based information extraction (IE) pipeline and corpus of 1,000 eligibility criteria documents for NER and relation extraction. The corpus was not made publicly available. Kang \textit{et al} created an annotation schema based on the Observational Medical Outcomes Partnership (OMOP) Common Data Model \cite{hripcsak2015observational} and an annotated corpus of 230 eligibility criteria documents, though the corpus focused narrowly on Alzheimer's Disease-related trials only \cite{kang2017eliie}. More recently, Kury \textit{et al} created Chia \cite{kury2020chia}, a publicly available corpus of 1,000 Phase IV trials similarly focused on the OMOP Common Data model but across a variety of disease domains and with a greater number of types of entities and relations. Yu \textit{et al} \cite{yu2020} released a corpus designed for direct text-to-query generation with semantic parsing, however given the relative simplicity of generated queries to date compared to the complexity of clinical databases, it is not clear this approach is yet viable for real-world clinical trials recruitment.

\subsection{Natural language processing approaches for cohort discovery}

Various methods for matching eligibility criteria to cohorts of patients using NLP have been put forth by the research community \cite{yuan2019criteria2query, soni2020patient, fang2022combining, zhang2020deepenroll, chen2019clinical, patrao2015recruit, dhayne2021emr2vec, liu2021evaluating, xiong2019cohort}. NLP-based cohort discovery methods hold unique potential and appeal, as they are theoretically able to leverage existing eligibility criteria described in natural language, a medium researchers and investigators already use and are comfortable with. Recent methods which utilize NLP in some form can generally be grouped into 5 categories:

\begin{enumerate}
    \item{\textbf{Database query generation} to Structured Query Language (SQL) or similar systems using either (a) rules, (b) neural network-based encoder-decoder architectures}, or both.
    \item{\textbf{Document ranking and classification} using clinical notes in terms of relevancy vis-à-vis a given eligibility criteria.}
    \item{\textbf{Projection into embeddings} of patient medical history and trial eligibility criteria in a shared vector space and matching via similarity measurement or entailment.}
    \item{\textbf{Logical representations and reasoning} to represent eligibility criteria and patient records, matching by combinations of Semantic Web technologies, ontologies, Description Logics, and rule-based reasoning.}
    \item{A combination of the above.}
\end{enumerate}

\noindent Next we briefly describe recent relevant work in each category. \\

\textbf{Database query generation} - SQL-based relational databases are widely used both commercially and within academic institutions, and as such SQL is perhaps unsurprisingly often used as a target language in natural language to database query research \cite{dar2019frameworks}. Yuan \textit{et al} developed Criteria2Query, a hybrid IE pipeline and application which uses both rules and machine learning to generate database queries on an OMOP database. This work was expanded by Fang \textit{et al}, who added functionality for iterative query generation via human correction and adjustment \cite{fang2022combining}. Although not specific to RCTs, other highly relevant recent work on query generation in the biomedical space has been done using encoder-decoder neural architectures for transforming clinical natural language questions into SQL queries \cite{bae2021question, park2021knowledge, wang2020text, pan2021bert, dhayne2021emr2vec}. Park \textit{et al} \cite{park2021knowledge} experimented with transforming medical questions generated in the MIMICSQL data set \cite{johnson2016mimic, wang2020text} using both SQL and SPARQL queries with varying database schema representations. Bae \textit{et al} similarly experimented with methods for handling typos, misspellings, and abbreviations in generating SQL queries from natural language questions. Pan \textit{et al} \cite{pan2021bert} leveraged intermediate abstract syntax tree-based representations and a SQL grammar-based decoder architecture for dynamic database schema matching. 

\textbf{Document ranking and classification} - Focusing on clinical notes, Chen \textit{et al} \cite{chen2019clinical} used hybrid rule-based heuristics and sentence pattern-matching to detect criteria structure, as well as a combination of neural network-based bi-directional long short-term and conditional random field (biLSTM+CRF) architecture and knowledge graphs using the Unified Medical Language System (UMLS) for determining condition, lab, procedure and drug relationships. Soni and Roberts \cite{soni2020patient} utilized the BERT Transformer architecture \cite{devlin2018bert} and Lucene \cite{lucene} to summarize, rank and classify clinical notes as relevant to a given eligibility criterion, with the most relevant notes predicted to be eligible. 

\textbf{Embedding projections} - Dhayne \textit{et al} \cite{dhayne2021emr2vec} experimented with treating patient-to-RCT matching as a joint embedding and similarity measurement problem while also incorporating the SNOMED-CT ontology to infer basic "is-a" and "has-type" relations between concepts. Similarly, Zhang \textit{et al} \cite{zhang2020deepenroll} used joint patient and eligibility criteria embeddings for entailment prediction, where predicting that a patient can be inferred from a given eligibility criteria equates to eligibility. 

\textbf{Logical representations and reasoning} - Patrao \textit{et al} developed Recruit \cite{patrao2015recruit}, an ontology-driven trial recruitment system which transformed SQL relational data to Resource Description Framework (RDF) graph-based triples. The RDF triples in turn were made query-able by use of an OWL-based reasoning system \cite{owl} and normalization techniques to infer cancer staging. Building upon earlier work \cite{patel2007matching, tu2009ergo, huang2013semanticct}, Baader \textit{et al} \cite{baader2018patient} explored the use of Description Logics and ontologies in matching patients in the MIMIC data set to logical representations of eligibility criteria, for example representing "Diabetes mellitus type 1" as "$\exists_y$.diagnosed\_with(x, y) $\wedge$ Diabetes\_mellitus\_type\_1(y)". Liu \textit{et al} \cite{liu2021evaluating} used domain experts to manually translate criteria into pseudo-code parsable by software, then executed as database queries to find patients meeting the criteria. For example, the criterion "Patients more than 18 years old when they received the treatment" would be represented as "\#Inclusion features['StartDate'] >= demographics['BirthDate'] + @YEARS(18)". 

\subsection{Interactive Applications for Cohort Discovery using NLP}

Systems allowing the querying of databases using natural language interfaces gained attention of the research community since the 1960s. Using relatively simple rule-based parsing systems, Woods \cite{woods1973progress} created a system for asking natural language questions of a moon rock database, while Epstein and Walker \cite{epstein1978natural} similarly designed a natural language interface for a melanoma database. Decades later, Katz \textit{et al} created START \cite{katz1999integrating}, a system capable of basic question-answering using data extracted and parsed from the internet. In the biomedical informatics domain, Cao \textit{et al} developed AskHERMES \cite{cao2011askhermes}, question-answering software capable of answering medical questions related to drugs, contraindications, and so on, using support vector machines (SVMs) and an internal knowledge base derived from the UMLS.

Cohort discovery systems using natural language are in many ways a subset of systems for question-answering which answer only one unstated question, "How many patients meet these criteria?". In the biomedical informatics and clinical trials domain, the most well-known and cited system for matching patients to clinical trials using free-text eligibility criteria is Criteria2Query \cite{yuan2019criteria2query, fang2022combining}. Criteria2Query offers a web-based simple and friendly user interface for inputting free-text eligibility criteria. The system analyzes user inputs and returns a highlighted listing of named entities it identified, as well as a potential list of patients meeting those criteria from an OMOP database.

\end{document}