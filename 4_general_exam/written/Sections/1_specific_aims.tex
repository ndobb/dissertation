\documentclass[../main.tex]{subfiles}

\begin{document}

\noindent Clinical trials serve a critical role in the generation of medical evidence and furthering of biomedical research. In order to identify potential participants, investigators publish eligibility criteria, such as past history of certain conditions, treatments, or laboratory tests. Patients meeting a trial's eligibility criteria are considered potential candidates for recruitment. Recruitment of participants remains, however, a major barrier to successful trial completion \cite{richesson2013electronic}, and manual chart review of hundreds or thousands of patients to determine a candidate pool can be prohibitively labor- and time-intensive. While cohort discovery tools such as Leaf \cite{dobbins2019leaf} or i2b2 \cite{murphy2010serving} can serve to assist in finding participants meeting eligibility criteria, such tools nonetheless often have significant learning curves, and certain complex queries may simply be impossible due to structural limitations on the types of possible queries presented.

An alternative approach which holds promise is the use of natural language processing (NLP) to automatically analyze eligibility criteria and generate queries to find patients in databases. NLP-based approaches have the advantage of obviating potential learning curves of tools such as Leaf, while leveraging existing eligibility criteria composed in a free-text format researchers are already familiar with.

The goal of this project is the development of an application called LeafAI, which if successful will enable the discovery of patients meeting criteria for clinical trials and general purpose biomedical research using natural language and generating queries for virtually any clinical database structure.

\subsection{Aim 1. Creation of Gold Standard Clinical Trials Data Set}
Using a random sample of clinical trials eligibility criteria from 2018 to 2021, we annotated over 1,000 documents for named entities and relations. Our annotation schema expanded on previous work \cite{kury2020chia, kang2017eliie} and introduced significantly greater semantic granularity in order to ease query generation and enable handling of particularly complex and non-specific criteria. Models trained on the resulting Leaf Clinical Trials (LCT) corpus achieved reasonably high named entity recognition and relation extraction performance of 81.3\% and 85.2\%. The resulting models enabled our subsequent work in Aim 2.

\subsection{Aim 2. Query Generation Method Development and Evaluation}
A successful application for cohort discovery using NLP for end-users hinges on development of robust, explainable methods for generating database queries. Thus in Aim 2 we focused on creation of novel approaches for cohort discovery query generation using NLP, including an integrated knowledge base (KB), Sequence to Sequence- (Seq2Seq) based logical transformations of eligibility criteria, normalization, and semantic metadata mapping (SMM) of database schema using concepts within the Unified Medical Language System (UMLS). To evaluate our system, we analyzed queries generated for 8 randomly chosen past clinical trials at our institution and compared enrolled patients to those predicted by LeafAI, the current state-of-the-art NLP-based system, Criteria2Query, and a human programmer.

\subsection{Aim 3. Web Application Development and Evaluation}
Aim 3 will focus on the development of a web application enabling automatic query generation using user-entered free-text eligibility criteria. The LeafAI web application will utilize a chat-like interface to enable iterative, explainable, interactive query generation. Users will then be able to edit and re-execute queries based on their findings. We will evaluate the effectiveness of the web application by comparing both (1) usability evaluations and (2) performance of generated queries in finding enrolled patients between LeafAI and Leaf.

\newpage

\end{document}